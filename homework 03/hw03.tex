\documentclass[12pt]{article}
\usepackage{amsfonts,amsmath,amsthm,amsbsy,graphicx,color,dsfont,listings, float, graphicx, amssymb, textcomp}

%
\renewcommand{\baselinestretch}{1}
\topmargin=-0.5truein
\textheight=9.0truein
\oddsidemargin=0.0truein
\textwidth=6.5truein
%
\pagestyle{empty}
\def \R{\mathbb R}
\def \E{\mathbb E}
\def \bmu{\boldsymbol \mu}
\def \P{\mathbb P}
\def \bX{\mbox{\boldmath$X$}}
\def \beps{\boldsymbol \epsilon}
\def \bx{\boldsymbol x}
\def \by{\boldsymbol y}
\def \bw{\boldsymbol w}
\def \bs{\boldsymbol s}
\def \bv{\boldsymbol v}
\def \bI{\boldsymbol I}
\def \bSigma{\boldsymbol \Sigma}
\def \btheta{\boldsymbol \theta}
\def \bH{\boldsymbol H}
\def \bmu{\boldsymbol \mu}
\def \1{\mathds{1}}
\def \F{{\cal F}}
\def \Rad{\mathfrak{R}}


\begin{document}

\begin{center} {\Large
{\bf Homework 3\\Shengchao Liu}}
\end{center}

\begin{enumerate}

% Problem 1
\item

$\alpha = \frac{1}{L}$, $\beta = \frac{\sqrt{L} - \sqrt{m}}{\sqrt{L} + \sqrt{m}}$.

Assume $u$ is eigenvlue, then we can get:

$u^2 - (1+\beta)(1-\alpha \lambda_i)u + \beta(1-\alpha \lambda_i)$,

thus $u = - \frac{1}{2} [(1+\beta)(1-\alpha\lambda_i) \pm \sqrt{(1+\beta)^2(1-\alpha\lambda_i)^2 - 4\beta(1-\alpha \lambda_i)} ]$

We need to prove that $(1+\beta)^2(1-\alpha\lambda_i)^2 - 4\beta(1-\alpha \lambda_i)<0$

Put $\alpha = \frac{1}{L}$, $\beta = \frac{\sqrt{L} - \sqrt{m}}{\sqrt{L} + \sqrt{m}}$ into formula, we can get:

$(1-\alpha \lambda_i)[(1+\frac{\sqrt{L} - \sqrt{m}}{\sqrt{L} + \sqrt{m}})^2(1-\frac{1}{L}\lambda_i) - 4\frac{\sqrt{L} - \sqrt{m}}{\sqrt{L} + \sqrt{m}}] $, and $1-\alpha \lambda_i>0$.

$ \Longrightarrow (1+\frac{\sqrt{L} - \sqrt{m}}{\sqrt{L} + \sqrt{m}})^2(1-\frac{\lambda_i}{L}) - 4\frac{\sqrt{L} - \sqrt{m}}{\sqrt{L} + \sqrt{m}} = \frac{4L}{(\sqrt{L} + \sqrt{m})^2} \frac{L-\lambda_i}{L} - \frac{4(L-m)}{(\sqrt{L} + \sqrt{m})^2} = \frac{4(m-\lambda_i)}{(\sqrt{L} + \sqrt{m})^2} < 0$

So the roots are distinct complex numbers.

\bigskip






% Problem 2

\item

$f(x) = \frac{1}{2} x^T Q x - b^T x + c$, $\bigtriangledown f(x) = Qx - bx$,

And $\bigtriangledown f(x^*)=0 \rightarrow Qx^* - b =0 \rightarrow x^* = Q^{-1}b$

Then we can write $\bigtriangledown f(x) = Qx - bx = Q(x-x^*)$

For heavy ball:

$x^{k+1} = x^k - \alpha \bigtriangledown(x^k) + \beta (x^k - x^{k-1})$,  where $\alpha = \frac{4}{(\sqrt{L} + \sqrt{m})^2}$ and $\beta = \frac{\sqrt{L} - \sqrt{m}}{\sqrt{L} + \sqrt{m}}$

$\Longrightarrow x^{k+1} - x^* = x^k - x^* - \alpha Q(x^k - x^*) + \beta ( (x^k-x^*) - (x^{k-1}-x^*)) $

$\begin{bmatrix} x^{k+1} - x^* \\ x^k - x^* \end{bmatrix} = \begin{bmatrix} I - \alpha Q + \beta & -\beta \\ I& 0\end{bmatrix}    \begin{bmatrix} x^k - x^* \\ x^{k-1} - x^* \end{bmatrix}$

By defining $w^k = \begin{bmatrix} x^{k+1} - x^* \\ x^k - x^* \end{bmatrix}$, $T = \begin{bmatrix} I - \alpha Q + \beta & -\beta \\ I& 0\end{bmatrix}$, we can have $w^k = T w^{k-1}$.

After rearranging the $T$ matrix, we can get $\begin{bmatrix} T_1  &  & & \\ & T_2 & & \\ & & \ddots & \\ & & & T_n\end{bmatrix}$, where $T_i = \begin{bmatrix} 1 - \alpha \lambda_i + \beta & -\beta \\ I& 0\end{bmatrix}$

And the eigenvalues of $T_i$ is solution to $u^2 - (1 - \alpha \lambda_i + \beta)u + \beta = 0 $.

$u = \frac{(1 - \alpha \lambda_i + \beta) \pm \sqrt{(1 - \alpha \lambda_i + \beta)^2 - 4 \beta}}{2}$

And then we can get:

\noindent$\begin{array}{lll}
(1 - \alpha \lambda_i + \beta)^2 - 4 \beta &=&(1 + \frac{L - m}{(\sqrt{L} + \sqrt{m})^2} - \frac{4\lambda_i}{(\sqrt{L} + \sqrt{m})^2})^2 - 4 \cdot \frac{L-m}{(\sqrt{L} + \sqrt{m})^2}\\
&=& (\frac{2L + 2\sqrt{Lm}}{(\sqrt{L} + \sqrt{m})^2} - \frac{4\lambda_i}{(\sqrt{L} + \sqrt{m})^2})^2 - 4 \cdot \frac{L-m}{(\sqrt{L} + \sqrt{m})^2}\\
&=& \frac{4}{(\sqrt{L} + \sqrt{m})^2} \cdot [(L+m-2\lambda_i)^2 - (L-m)(L+m+2\sqrt{LM})] \\
&=& \frac{4}{(\sqrt{L} + \sqrt{m})^2} \cdot [m^2 + Lm + 2m\sqrt{Lm} + 4\lambda_i^2 - 4L\lambda_i -4\lambda_i\sqrt{Lm}]
\end{array}$

Then we set $\phi(\lambda) = 4 \lambda^2 -4(L+\sqrt{Lm})\lambda + m^2 +Lm + 2m\sqrt{Lm}$, and can easily get $\phi(L) < 0$, and $\phi(m) < 0$

Combined with the fact that $\phi(\lambda)$ is quadratic, $\phi(\lambda) < 0$, $\forall \lambda \in [m, L]$, so $(1 - \alpha \lambda_i + \beta)^2 - 4 \beta < 0$, which means the $u_{i,1}$ and $u_{i,2}$ are two complex numbers.

$u_{i,1} = \frac{1}{2} [(1 - \alpha \lambda_i + \beta) + i \sqrt{4 \beta - (1 - \alpha \lambda_i + \beta)^2}] $

$u_{i,2} = \frac{1}{2} [(1 - \alpha \lambda_i + \beta) - i \sqrt{4 \beta - (1 - \alpha \lambda_i + \beta)^2}] $

$| u_{i,1} | = | u_{i,2} | = \frac{1}{2} \sqrt{(1 - \alpha \lambda_i + \beta)^2 + 4 \beta - (1 - \alpha \lambda_i + \beta)^2 } = \sqrt\beta $

So $\rho(T) = \sqrt\beta = \sqrt\frac{\sqrt{L} - \sqrt{m}}{\sqrt{L} + \sqrt{m}}$

And according to the Gelfand's Formula, we can get

$\rho(T) = (\underset{k\to\infty}{lim}\|T^k\|)^{{1}/{k}}$

A consequence is that $\forall \epsilon > 0$, there is $\|T^k\| \le c(\rho(T) + \epsilon )^k$

$\Longrightarrow \|w^k\| = \|T^k w^0\| \le \|T^k\| \|w^0\| \le c \|w^0\| (\rho(T) + \epsilon )^k$

And because $\rho(T) = \sqrt\frac{\sqrt{L} - \sqrt{m}}{\sqrt{L} + \sqrt{m}} < 1$, so $\|w^k\|$ converges linearly to zero.

\bigskip











% Problem 3

\item

According to problem, we have $x^0=0$, and $x_i^* = 1 - i / (n+1)$,

$\| x^0 - x^*\|^2_2 = \sum_{i=1}^n (1-\frac{i}{n+1})^2 = \frac{1}{(n+1)^2} \cdot \frac{n(n+1)(2n+1)}{6} \le \frac{n \cdot 2(n+1)}{6(n+1)} = \frac{n}{3}$

$\| x^k - x^* \|^2_2 = \sum_{i=k+1}^n (1-\frac{i}{n+1})^2 = \frac{1}{(n+1)^2} \cdot \frac{(n-k)(n-k+1)(2n-2k+1)}{6} \ge \frac{(n-k)\cdot (n-k) \cdot 2(n-k)}{6(n+1)^2} = \frac{(n-k)^3}{3(n+1)^2}$

And because $\frac{1}{3} \ge \frac{\| x^0 - x^*\|^2_2}{n}$, we get $\| x^k - x^* \|^2_2 \ge \frac{(n-k)^3}{3(n+1)^2} \ge \frac{(n-k)^3}{n(n+1)^2}\| x^0 - x^*\|^2_2$

\end{enumerate}

\end{document}

